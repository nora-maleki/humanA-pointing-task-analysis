\chapter{Introduction}


Spatial navigation entails the orientation and movement planning in an area. In the field of Cognitive neuroscience, it is investigated how space, "the abstract of all co-existence" \autocite{spencer1989}, is processed by the brain, memorized, and retrieved for navigation. 

Over the past centuries, theories were developed about how space and the relation to its objects, thus distance, is represented in the brain, e.g., \textcite{cassirer1955philosophy} categorizes spatial knowledge in three different temporal levels: first a close encounter with the objects and their spatial relation to one another that allows for interaction with them. The second level contains a wider space that allows for building the relationship between routes and mental maps. Realizing the relations of the places, routes, etc using the symbolic system to represent a space is the last level \autocite{cassirer1955philosophy}. 

Performing studies outside of the laboratory condition could have difficulties regarding control of the surrounding environment, replicability and ecological validity. VR is a reachable solution for these problems \autocite{pan2018and, diersch2019potential, chicchi2017novel}. VR offers a range of advantages compared to conducting the experiment in the real world. From VR experiments, in addition to the classical behavioral data such as response times, a broader range of variables can be recorded, e.g., movement of the body, hands, head, and eyes \autocite{pan2018and}. Needless to say that these variables can be gathered in a controlled environment when utilizing VR as the method of measurement \autocite{mcilvenny2020future}. These can be also very beneficial for spatial navigation studies.

Human beings are born and grow up in social environments. They take the social aspects of their surroundings as their own \autocite{berger1967luckman}. They interact with society and interpret reality by what their culture is constructed of \autocite{SIEGEL19759}. These aspects are also applicable in spatial navigation. \textcite{kuehn2018social} shows that the social components can even be more powerful factors for encoding space when their participants perform consistently more accurately as they have to guess the position of the human agent as the target in comparison to the position of an object.

The present work is a small part of the Human-A project, a Ph.D. project built and executed at the Neuroinformatics department of the institute of Cognitive Science at the University of Osnabrück. Human-A is a spatial navigation experiment in a small European fictive VR city. The study conducted in this environment aims more at the social aspects and their effects on learning the space by using the functionality of some buildings, i.e., the social meaning of shops, e.g., bakery, bookstore, and adding human agents to the environment. The participants get the chance to explore and learn about the city before the test session. Testing is a pointing task performed from different locations in the city, showing photos of buildings from the city and asking them to point towards them. The behavioral data gathered from the tasks consist of the angular deviation of the participant's chosen direction from the actual target location, and reaction times data.

The Human-A study leans toward the social factors of spatial navigation, hence it is essential for validity to check whether the other existing factors in the experiment could be confounding. One of those factors is the different locations from which the participants perform the tasks, i.e., the different starting locations. 

Investigating the effect of the social aspects of the environment entails having built a mental map and knowing the city, then it is hypothesized in this thesis that the change in starting locations has no effect on the angular deviation from the target building and as well has no effect on reaction times.

\newpage