\markboth{}{}

\begin{minipage}[l]{80mm}
	\vspace{120pt} 
	\leftskip=5em
	Imagine a long blackout, the eyes are closed and there is no sound. The consciousness is slowly coming back and the question arises: where am I?\\
\end{minipage}


\vspace{20pt}


\chapter{Introduction}


Spatial navigation entails the orientation and movement planing in an area. In the field of Cognitive neuroscience it is investigated how space, "the abstract of all co-existence" \autocite{spencer1989}, is processed by the brain, memorized and retrieved for navigation. 

Over the past centuries theories were developed about how space and the relation to its objects, thus distance, is represented in the brain, e.g., \textcite{cassirer1955philosophy} categorized spatial knowledge in three different temporal levels, a close encounter with the objects and their spatial relation to one another allows for interaction with them. The second level contains a wider space that that allows for building the relation of routes and mental maps. The final step is realizing the relations of the places, routes, etc using the symbolic system to represent an space. 

Performing a spatial navigation study in the real world has some known difficulties such as not being able to control the surrounding environment, thus having confounding variables, and not being able to or it being very hard to gather usable data for the analysis. VR is a reachable solution for these problems \autocite{diersch2019potential}. VR offers a range of advantages compared to conducting spatial navigation in real-world. From VR experiments in addition to the classical behavioral data, e.g., response times, a broader range of variables can be recorded, e.g., movement of the body, hands, head, eyes \autocite{pan2018and}. Needless to say that these variables can be gathered in a controlled environment when utilizing VR as the method of measurement \autocite{mcilvenny2020future}. Apart from these VR is shown to offer ecological validity \autocite{pan2018and, chicchi2017novel} and reproducibility \autocite{pan2018and}.

Human beings are born and grow up in social environments. They take the social aspects of their surroundings as their own \autocite{berger1967luckman}. They interact with the society and interpret the reality by what their culture is constructed of \autocite{SIEGEL19759}. These aspects are also applicable in spatial navigation. \textcite{kuehn2018social} shows that the social components can even be more powerful factors for encoding space when their participants perform consistently more accurate as they have to guess the position of the human agent as target in comparison to the position of an object.

Human-A, the present study, is a spatial navigation experiment built in a small European fictive VR city. The study conducted in this environment aims more at the social aspects and their affects on learning the space by using the functionality of some buildings, i.e., the social meaning of shops, e.g., bakery, bookstore, and adding human agents to the environment. The participants get the chance to explore and learn the city before the test session. Testing is a pointing task performed from different locations in the city, showing photos of buildings from the city and asking them to point towards them. The behavioral data gathered from the tasks consist of the angular deviation of participant's responded direction from the actual target location, and reaction times data.

The Human-A study leans toward the social factors of spatial navigation, hence it is essential for validity to check whether the other existing factors in the experiment could be confounding. One of those factors is the different locations from which the participant's perform the tasks, i.e., the different starting locations. 

Investigating the effect of the social aspects of the environment entails having built a mental map and knowing the city, then it is hypothesized in this thesis that the change in starting locations has no effect on the angular deviation from the target building and as well has no effect on reaction times.
