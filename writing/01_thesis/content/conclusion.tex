\chapter{Conclusion}

The work of this thesis was based on the question of whether or not the task's starting locations affected the two dependent variables targeted in this analysis, i.e., the absolute angular deviation and the reaction times.

As for the best and worst locations, No. 9 and No. 35 respectively, it seemed that being located inside or on the edges of the city could affect the outcome. At location No. 9, it is possible that a very large portion of the angular possibilities are eliminated due to the fact that if the participant sees that there is no part of the city on one side of where they are, it will be highly unlikely to point there.

Furthermore, due to this elimination, there would be also a higher likelihood of pointing by chance and making small errors. Similarly for the starting location No. 35, due to it being about the center of the city, it is more likely to get widely located targets around the starting location, and therefore the error rate could rise. This could also be one of the explanations for location 35 being the slowest starting location. The spread of the targets could cause the participants the need to think more accurately about the target. 

The results of the LMM, absolute angular deviation as a function of starting location, seemed to show that almost half of the starting locations affect the outcome significantly. There are some factors that should be taken into consideration for this outcome. There seems to be a slight trend of having a better performance in the starting locations that lie on the edge of the city and their view is not blocked by buildings.

Looking at the results of predicting RT by starting locations, clearly, the minority of the starting locations affect the reaction times significantly. From those, it is also possible that the locations being located on the edges of the city or inside make the difference. 

In conclusion, this thesis cannot reject the null hypothesis that the starting locations do not have an effect on the absolute angular deviation and reaction times for all the locations. Based on the residuals of the models which are not completely normally distributed, it can be concluded that there are other factors as well affecting the outcome. Other factors such as distance to target and context (meaningfulness) of the starting locations, and their interactions with the starting locations should also be taken into account to find the best model for explaining the variance. Furthermore, the independent variables trial id per starting location which contains the 12 consecutive trials performed from each starting location can be also a factor to investigate.

