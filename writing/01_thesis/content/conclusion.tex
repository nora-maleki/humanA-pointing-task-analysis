\chapter{Conclusion}

Work of this thesis was based on the question whether or not the task's starting locations effect the two dependent variables targeted at this analysis, i.e., the absolute angular deviation and the reaction times.

As for the best and worst locations, No. 9 and No. 35 respectively, it seems that maybe being located inside or on the edges of the city could be a factor of this outcome. At the location No. 9, it is possible that a very large portion of the angular possibilities are eliminated due to the fact that if the participant sees that there is no part of the city on one side of where they are, it will be highly unlikely to point there.

 Furthermore, due to this elimination, there would be also a higher likelihood of pointing by chance and making small errors. Likely for the starting location No. 35, due to it being about the center of the city, it is more likely to get widely located targets around the starting location and therefore the error rate could rise. This could also be one of the explanation for location 35 being the slowest starting location. The spread of the targets could cause the participants the need to think more accurately about the target. 

The results of the LMM, absolute angular deviation as a function of starting location, seems to show that almost half of the starting locations effect the outcome significantly. There are some factors that should be taken into consideration for this outcome. There seems to be a slight trend of having a better performance by the starting locations that lie on the edge of the city and their view is not blocked by buildings.
Looking at the results of predicting RT by starting locations, clearly the minority of the starting locations affect the reaction times significantly. From those it is also possible that the locations being located on the edges of the city or inside make the difference. 

Moreover, the results of LMM were not able to explain the variance of the data finely. It appears that other factors existing in the experiment should also be taken into account to find the best model for explaining the variance. Factors such as distance to target and meaningfulness of the location should be also investigated to maybe be able to answer the outcome. Furthermore, the independent variables trial id per starting location which contains the 12 consecutive trials performed from each starting location can be also a factor to investigate.\\


\iffalse
\todo{
	- in case there are differences found between starting locations discuss the reasons \\
	- if there are previous studies, compare the results \\
	- discuss limitations and suggest improvements if needed\\
	- what can be done further on this topic \\
	- summarize conclusion \\
	- looking into trials at each starting location. the 12 as further analysis
}

\fi