\chapter*{Abstract}\label{cha:abstract}


This thesis is investigating the effect of starting locations from which a pointing task in a spatial navigation study inside a virtual reality (VR) city was performed. The experiment consisted of two parts, i.e., exploration and testing. In the exploration part, the participants were instructed to freely explore the city. After five exploration sessions, a testing session took place where the participants were put in 28 different locations in the city, and from each, they had to perform 12 trials. They were shown a photo of a building inside the city. The task was to indicate the direction of the building.

The experiment is a Ph.D. project designed mainly with a social context, realized by placing human agents inside the city in front of context meaningful and non-meaningful buildings. Two main dependent variables were measured in this study, i.e., the absolute angular deviation from the target building and the reaction times. In order to investigate those factors, it was important to have knowledge about the effect of the locations the tasks are performed. This led to this work.

In this work, the effect of the starting locations on the absolute angular deviation from the target and the reaction times were analyzed with linear mixed models. The results showed significant effects of some locations but not others on the dependent variables. It appeared that the difference could be due to the placement of the different locations in the city. The ones that were closer to the edges had better performances than the ones in the middle of the city. This could be caused by the opportunity to eliminate some directions if it was clear that the city does not exist on a side of a location. Furthermore, it seemed that there could be also other factors or interactions of starting locations with other factors which might better explain the data.

\newpage
\shipout\null